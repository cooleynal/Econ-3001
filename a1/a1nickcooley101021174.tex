\documentclass[12pt]{article}
\usepackage{fullpage}
\usepackage{fancybox}
\usepackage{hyperref}
\usepackage{amsfonts}
\usepackage{graphicx}
\usepackage{amsmath}
\usepackage{xcolor}
\usepackage{enumitem}
\newcommand{\SubSum}{\textsf{\sc SubsetSum}}
\newcommand{\IP}{\textsf{\sc IntProg}}
\newcommand{\SAT}{\textsf{\sc Sat}}
\newcommand{\DNFSAT}{\textsf{\sc DNFSAT}}
\newcommand{\HC}{\textsf{\sc HamiltonCycle}}
\newcommand{\SC}{\textsf{\sc SetCover}}

\newcounter{ques}
\newenvironment{question}{\stepcounter{ques}{\noindent\bf Question \arabic{ques}:}}{\vspace{5mm}}

\begin{document}

\begin{center} \Large\bf
      Econ 3001B - Winter 2023
      Name: Nick Cooley.
      Student number: 101021174.
\end{center}

\noindent {\bf Due:} 01 March 2023

\vspace{0.5em}

% \newpage

\color{blue}
\color{black}


\begin{question}

\vspace{0.5em}



\begin{enumerate}[label=(\alph*)]
      \item $$
      \begin{bmatrix} 
      27 & 44 & 51   \\
      35 & 39 & 62   \\
      33 & 50 & 47   \\
      \end{bmatrix}
      +
      \begin{bmatrix} 
      25 & 42 & 48 \\
      33 & 40 & 66\\
      35 & 48 & 50 \\
      \end{bmatrix}
      \quad
      $$
      $$
      \begin{bmatrix} 
            27 + 25 & 44 +42 & 51 + 48   \\
            35 + 33 & 39 + 40 & 62 + 66   \\
            33 + 35 & 50 + 48 & 47 + 50   \\
      \end{bmatrix}
      \quad
$$


$$
\begin{bmatrix} 
      52 & 86 & 99 \\
      68 & 79 & 128\\
      68 & 98 & 97 \\
      \end{bmatrix}
\quad
$$

      \item Solve AB and BA, where
  
      $$
      A = 
      \begin{bmatrix} 
      1 & 2 \\
      2 & 1
      \end{bmatrix}
      , B = 
      \begin{bmatrix} 
      1 & 0 \\
      1 & 0
      \end{bmatrix}
      \quad
$$

$$
AB =
\begin{bmatrix} 
1*1 + 2*1 & 1*0 + 2*0 \\
2*1 + 1* 1 & 2*0 + 1*0
\end{bmatrix}
\quad
$$


$$
AB =
\begin{bmatrix} 
3 & 0 \\
3 & 0
\end{bmatrix}
\quad
$$

$$
BA =
\begin{bmatrix} 
1*1 + 0*2 & 1 * 2 + 0*1 \\
1*1 + 0*2 &1*2 + 0*1
\end{bmatrix}
\quad
$$


$$
BA =
\begin{bmatrix} 
1 & 2  \\
1 & 2
\end{bmatrix}
\quad
$$


In general, matrix multiplication is noncommutative as $BA \neq AB.$


      \item Compute $(A+B)^T$, for A and B below:
      


      $$
      A = 
      \begin{bmatrix} 
      1 & 2 \\
      3 & 0
      \end{bmatrix}
      , B = 
      \begin{bmatrix} 
      3 & 1 \\
      -1 & 1
      \end{bmatrix}
      \quad
$$


$$
A + B = 
\begin{bmatrix} 
1 + 3 & 2 + 1 \\
3 + -1 & 0 + 1
\end{bmatrix}
\quad
$$

$$
A + B = 
\begin{bmatrix} 
4 & 3 \\
2 & 1
\end{bmatrix}
\quad
$$

$$
(A + B)^T = 
\begin{bmatrix} 
4 & 2 \\
3 & 1
\end{bmatrix}
\quad
$$

$$
(A + B)^T - B^T= 
\begin{bmatrix} 
4 & 2 \\
3 & 1
\end{bmatrix}
-
\begin{bmatrix} 
3 & 1 \\
-1 & 1
\end{bmatrix}
\quad
$$

$$
A^T + B^T - B^T= 
\begin{bmatrix} 
1 & 3 \\
2 & 0
\end{bmatrix}
+
\begin{bmatrix} 
      3 & -1 \\
      1 & 1
      \end{bmatrix}
-
\begin{bmatrix} 
3 & -1 \\
1 & 1
\end{bmatrix}
\quad
$$

$$
A^T + B^T - B^T= 
\begin{bmatrix} 
1 & 3 \\
2 & 0
\end{bmatrix}
+
\begin{bmatrix} 
      0 & 0 \\
      0 & 0
      \end{bmatrix}
\quad
$$
Additive inverse on $M(\mathbb{R})_{2x2}$ yields:
$$
A^T = 
\begin{bmatrix} 
1 & 3 \\
2 & 0
\end{bmatrix}
\quad
$$

and,
$$
B^T = 
\begin{bmatrix} 
3 & -1 \\
1 & 1
\end{bmatrix}
\quad
$$

Thus: $A^T + B^T = (A + B)^T$ which was to be shown.

\end{enumerate}
\end{question}

\begin{question}
 

\begin{enumerate}[label=(\alph*)]
       
      \item Compute the following limits:
      

      \begin{enumerate}[label=(\alph*)]
       
            \item $\lim_{x \to -2}(x^2 + 5x)$
            
            $\lim_{x \to -2^+}(x^2 + 5x)$ 

            $$
            ((2^+)^2 + 5(2^+))
            $$

            $$
            (4 + 10)
            $$

            $$
            \lim_{x \to -2^+}(x^2 + 5x) = 14
            $$

            $\lim_{x \to -2^-}(x^2 + 5x)$

            $$
            ((2^-)^2 + 5(2^-))
            $$
            
            $$
            (4+ 10)
            $$

            $$
            \lim_{x \to -2^-}(x^2 + 5x) = 14
            $$


            $$\lim_{x \to -2^-}(x^2 + 5x) = 14 = \lim_{x \to -2^+}(x^2 + 5x) $$

            $$\lim_{x \to -2}(x^2 + 5x) = 14$$


            \item $\lim_{x \to 4} \frac{2x^{\frac{3}{2}} - x^{\frac{1}{2}}}{x^2 - 15}$
            
            $$
            \lim_{x \to 4^+} \frac{2x^{\frac{3}{2}} - x^{\frac{1}{2}}}{x^2 - 15}
            $$

            $$
            \frac{2(2^+))^{\frac{3}{2}} - (4^+)^{\frac{1}{2}}}{x^2 - 15}
            $$


            $$
            \frac{2(2^+)(2^+)(2^+) - (2^+)}{(16^+) - 15}
            $$

            $$
            \frac{16 - 2}{1}
            $$

            $$
            \lim_{x \to 4^+} \frac{2x^{\frac{3}{2}} - x^{\frac{1}{2}}}{x^2 - 15} = 14
            $$

            $$
            \lim_{x \to 4^-} \frac{2x^{\frac{3}{2}} - x^{\frac{1}{2}}}{x^2 - 15}
            $$

            $$
            \frac{2(2^-)(2^-)(2^-) - (2^-)}{(16^-) - 15}
            $$

            $$
            \frac{16 - 2}{1}
            $$

            $$
            \lim_{x \to 4^+} \frac{2x^{\frac{3}{2}} - x^{\frac{1}{2}}}{x^2 - 15} = 14 = 
            \lim_{x \to 4^-} \frac{2x^{\frac{3}{2}} - x^{\frac{1}{2}}}{x^2 - 15}
            $$

            $$\lim_{x \to 4} \frac{2x^{\frac{3}{2}} - x^{\frac{1}{2}}}{x^2 - 15} = 14$$
      
            \item $\lim_{x \to a}(A x^n)$
            
            We need to be careful, as the following does not hold for $ \forall [x, a, n] \in \mathbb{R}$, for example 
            if $a = 0$ and $n < 0$ then we are dividing by zero which is undefined as the left sided and right sided limits diverge from each other.

            Asumming $ x, a \geq 0$.
            $$\lim_{x \to a+}(A x^n)$$

            $$\lim_{x \to a+}(A x^n) = A (a^+)^n$$


            $$ = (A x^n) $$

            Asumming $ x, a > 0$.
            $$\lim_{x \to a^-}(A x^n)$$

            $$\lim_{x \to a^-}(A x^n) = A (a^-)^n$$

            $$ = (A x^n) $$

            $$\lim_{x \to a^-}(A x^n) = A a^n = \lim_{x \to a+}(A x^n)$$

            $$\lim_{x \to a}(A x^n) = A a^n$$ If and only if $ x, a > 0, $ if $n < 0$ otherwise if $n > 0$ then $x, a$ are free and the limits converge.

      \end{enumerate}


      \item Find an expression for $dz$ in terms of $dx$ and $dy$ in the following:
      \begin{enumerate}[label=(\alph*)]
            \item $z = Ax^a + By^b$
            $$z = f(x,y) = Ax^a + By^b$$

            $$ dz = \frac{\partial f}{\partial x} dx + \frac{\partial f}{\partial y} dy$$

            $$ dz = aAx^{a-1} dx +  bBy^{b-1} dy$$


            \item $z = e^{xu}, where \ u = u(x,y).$
            
            $$z = f(x,y) = e^{xu}, where \ u = u(x,y)$$

            % Label $g(x,u) = e^{xu}$.

            $$ dz = \frac{\partial f}{\partial x} dx +
             \frac{\partial f}{\partial u} \frac{\partial u}{\partial x}dx +
             \frac{\partial f}{\partial u} \frac{\partial u}{\partial y} dy
            $$

      
            $$ dz =  \left(\frac{\partial f}{\partial x}  +
            \frac{\partial f}{\partial u} \frac{\partial u}{\partial x} \right)dx +
             \frac{\partial f}{\partial u} \frac{\partial u}{\partial y} dy 
            $$

            $$ dz =  \left(ue^{xu} +
            xe^{xu}  \frac{\partial u}{\partial x} \right)dx +
            xe^{xu} \frac{\partial u}{\partial y} dy 
            $$

            $u = u(x,y)$ is unknown so its derivative with respect to x and y are unknown.
            \item $z = ln(x^2 + y)$
            

            $$z = f(x,y) = ln(x^2 + y)$$

            let $u = x^2 + y$
            $$ dz = \frac{\partial f}{\partial u}\frac{\partial u}{\partial x} dx +
            \frac{\partial f}{\partial u}\frac{\partial u}{\partial y} dy
            $$

            $$ dz = \frac{\partial f}{\partial u} \left( \frac{\partial u}{\partial x} dx +
            \frac{\partial u}{\partial y} dy\right)
            $$

            $$ dz = \frac{1}{u} \left( 2x dx +
            dy\right)
            $$

            $$ dz = \frac{2x + 1}{x^2 + y}
            $$


      \end{enumerate}
\end{enumerate}

\end{question}



\begin{question}

\end{question}
Find $A^{-1}$
$$
A = 
\begin{bmatrix} 
1 & 2 & 3 \\
0 & 1 & -1 \\
1 & 2 & 1
\end{bmatrix}
\quad
$$


$$det(A) = (1 * 1 * 1 )+ (2 * -1 *1) + (3* 0 * 2) - (2*0*1) - (1 *-1 * 2) - (3* 1 * 1)$$

$$det(A) = 1+ -2 + 0 - 0 - -2 - 3 = -2$$

Matrix is invertible.

$$A^{-1} = \frac{1}{det(A)} cof(A)^T = \frac{1}{det(A)} adj(A)$$

$$
minidets(A) = 
\begin{bmatrix} 
3 & 1 & -1 \\
-4 & -2 & 0 \\
-5 & -1 & 1
\end{bmatrix}
\quad
$$

$$
cof(A) = 
\begin{bmatrix} 
3 & -1 & -1 \\
4 & -2 & 0 \\
-5 & 1 & 1
\end{bmatrix}
\quad
$$



$$
cof(A)^T = adjoint(A) = 
\begin{bmatrix} 
3 & 4 & -5 \\
-1 & -2 & 1 \\
-1 & 0 & 1
\end{bmatrix}
\quad
$$

$$
A^{-1} = 
\frac{1}{-2}
\begin{bmatrix} 
3 & 4 & -5 \\
-1 & -2 & 1 \\
-1 & 0 & 1
\end{bmatrix}
\quad
$$

$$
A^{-1} = 
\begin{bmatrix} 
\frac{-3}{2} & -2 & \frac{5}{2} \\
\frac{1}{2} & 1 & \frac{-1}{2} \\
\frac{1}{2} & 0 & \frac{-1}{2}
\end{bmatrix}
\quad
$$


\begin{question}
      Consider the National -Income model with 3 endogenous variables, Y (national income), C (consumption), and t (taxes).

      $$Q_d = a-bp \quad (a, b> 0)$$
      $$Q_s = -c +dp \quad (c, d > 0)$$

      Endogenous variables are:  $\{P, Q\}$ which are functions of the exogenous variables: $\{a,b,c, d\}$.
     


      \begin{enumerate}[label=(\alph*)]
            \item Derive $P^*$ and $Q^*$ in equilibrium (when quantity supplied = to quantity demanded)
      
            We want to find the intersection of the supply and demand curves.
            $$D = \{(P,Q) | Q = a-bP\} $$
            $$S = \{(P,Q) | Q -c +dP\} $$
            $$D \cap S = (P^*, Q^*)$$
            $$  Q_d = a-bP = -c +dP = Q_s \quad (c, d > 0), (a, b > 0)$$
            $$   a + c = bP +dP  $$
            $$   a + c = P(b +d)  $$
            $$   \frac{a + c}{b +d} = P^* $$
            $$ Q^* = a-b\left(\frac{a + c}{b +d}\right) $$
            $$ Q^* = \frac{a(b + d) - b(a + c)}{b +d} $$
            $$ Q^* = \frac{ab +ad - ba - bc}{b +d} $$
            $$ Q^* = \frac{ab - ab +ad - bc}{b +d} $$
            $$ Q^* = \frac{ad - bc}{b +d} \quad (a, b, c, d > 0)$$
            $$   P^* = \frac{a + c}{b +d}  \quad (a, b, c, d > 0)$$

            \item Examine the comparative-static properties of the equilibrium quantity and provide the economic meaning
            of it? (Note compute partial derivatives 
            of $P^*$ with respect to parameters in the model. We discuss this in details in class during lecture)

            This comparative-static model reflects the equilibrium point $(Q^*, P^*)$ with respect to a single commodity. 
            The exogenous variables $a$ and $c$ reflect the Q axis intercept, while the remaining exogenous variables $b$ and $d$ 
            reflect the change in quantity with respect to $P$.

      \end{enumerate}


\end{question}

\end{document}